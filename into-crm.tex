\documentclass[a4paper,12pt]{article}
\usepackage{ucs}
\usepackage[utf8]{inputenc}
\usepackage{babel}
\usepackage{fontenc}
\usepackage{graphicx}
\usepackage{hyperref}
\usepackage{caption}
\usepackage{float}


\begin{document}

    \begin{titlepage}
        \begin{center}
            \vspace*{1cm}

            \textbf{Into Research Methods in Computer Science}

            \vspace{0.5cm}
                A Bit of First Things Firstness
                    
            \vspace{1.5cm}

            \textbf{Ivan Ryansyach*}

            \vfill
                    
            \vspace{0.8cm}
            
            Faculty of Engineering\\
            The State University of Makassar\\
            Indonesia\\
            4 September 2021
                    
        \end{center}
    \end{titlepage}

    \newpage
    \begin{abstract}
        This article can serve as a pilot course for undergraduates who are getting started on research, especially in the field of computer science. Topics that are covered here aimed to be easily understood in a single reading session, therefore some terms that are deemed too technical will be avoided as much as possible. This is not an official project and merely a continuation of author's own previous work. Original article is submitted as part of class assignment in Computing Research Methodology lecture in the Faculty of Engineering, The State University of Makassar.
    \end{abstract}

    \newpage
    \tableofcontents

    \newpage
    \section{Research Landscape in Computer Science}
        \subsection{What constitutes a computer science research?}
            According to the ACM Computer Science Body of Knowledge, this area of study generally revolves around these 3 things:
            \begin{itemize}
                \item Designing and implementing software
                \item Engineering new approach of using computer
                \item Creating more effective solution to solve computing problems
            \end{itemize}

            While such focuses are what made Computer Science (CS) that we know today, it is also important to know, that research carried out in this particular field may sometimes fall into a more broader, less specific subject area. Since the power of computing has raised the current global societies to a higher standard, by offering affordable alternative to plethora of manual labor. All thanks to its deep root in both Mathematics and Technology, the practical usage of CS finds its way in a multidiscplinary environment through relative ease.

            Getting started in CS research can be confusing and intimidating, especially for undergraduate students, this is a big issue. After all, research per se is an investigation and study on top of a source material to find new facts\cite{Utami07}. To advance the CS knowledge boundary and produce a good research result we will need a systematical approach. And to that end, this writing is aimed as a quick start guide on Computing Research Methods (CRM). Throughout this article we will explore some undergoing major topics in CS research, and provide overview of commonly used research methods by computer scientists throughout the decade. Hopefully, to give our dear reader some early concepts and basic knowledge that can be improved later on as the time goes.
        
        \subsection{Examples from current issues}
            Below, are some of the ongoing research topics in the CS field\cite{Ramesh02}. Some of these are CS-specific, while others are multidiscplinary. Notice how each topic are unique and may have developed its own approach of research method.
            \begin{itemize}
                \item Mathematics/computational science (artificial intelligence, machine learning, big data)
                \item Software engineering (requirements, design, construction, testing, and maintenance)
                \item Scientific/engineering (bio-informatics, chemo-informatics, robotics, telemedicine)
                \item Project management (process management, risk management, and personnel issues)
                \item Technology transfer (innovation, acceptance, adoption, diffusion)
                \item Societal impacts of computer science (cultural, legal, ethical, political)
            \end{itemize}

    \section{Overview of Common Research Methods}
        Though there is no universally agreed approach, CRM usually falls into one out of these five category. Also, while some may have defined six or seven different approaches, and others may go as far as ten, this shouldn't really be a problem. As the broadness of CS itself covers a lot of things, the main focus should always be the research questions or problems rather than the difference in research methods.

        \subsection{Implementation approach}
            A way of tackling current issues by creating or building a new software. It can be a high impact research when the software is used by many. The complexity of implementation process should be considered as well from the beginning of the research. Because of the time needed for construction and varieties of implementation techniques, which oftentimes puts research at disarray.

        \subsection{Observational studies}
            An approach that's done under the pretext of collecting data gathered from certain system(s) under a given circumstance. The goal is always to determine how such systems perform in real world case. The data is then served as a basis for devising a new way of countering current problems or limitations.
            This approach is usually done in a repeating cycle, from devising a plan, implement the plan, observe, reflect, revise a new plan, implement the new plan and so on and so forth.

        \subsection{Literature surveys}
            A way of solving research questions through thorough mapping of what is currently known and what is currently open to question. It is a strategy that yields a complete, and clearly defined scope of research. A rigorous method that's achieved by exhaustive comparison from lots of previous study or published paper concerning the same issue preferably with high impact rating.

        \subsection{Empirical methods}
            An empirical studies originates from experiences or direct observations of certain phenomenon, which in turn leads to a new hypothesis. This new hypothesis is then tested in a carefully designed experiment supplied with strong statistical support to provide solid evidence. The quality of empirical studies is reflected by the similiarity of the same subsequent experiment. The more replicable it is, the better the quality.

        \subsection{Formal methods and simulations}
            When the circumstance proves dangerous or incredibly expensive to conduct a direct experiment, a research based on formal methods or simulations may be performed. While formal methods aim for an indisputable answer through solid mathematical proof, a simulations ought to represent an abstraction of a real world object, with the necessary property of object in question remains intact to allow a higher degree of experiment control.

    \section{Summary and Extras}
        To summarise this article, here are four things that's good to know for those who are diving into research for the first time\cite{Shapiro15}.

        \subsection{Newcomer's corner}
            This section lists some tips that may provide useful in both before and during an ongoing research.

            \subsubsection{Preparation to move forward}
                \begin{itemize}
                    \item Have a clear goal in mind. And keep at it.
                    \item Improve your current knowledge and technical understanding.
                    Also, published study is what's known a year ago or more, and but talking in seminars and to fellow researchers requires up-to-date knowledge.
                \end{itemize}

            \subsubsection{The next stage}
                \begin{itemize}
                    \item Great discoveries are always backed by good documentation.
                    \item Time management for maximum productivity.
                \end{itemize}

    \begin{thebibliography}{100}
        \bibitem{Utami07} Utami, Istyanto, Raharjo. (2007). \emph{Metodologi Penelitian Pada Ilmu Komputer}, Seminar Nasional Teknologi. \url{https://www.academia.edu/download/50533733/Handout-2_Metode_Penelitian_Pada_Ilmu_Komputer.pdf}

        \bibitem{Ramesh02} Ramesh, Glass, Vessey. (2002). \emph{Research in Computer Science: An Empirical Study}, Journal of Systems and Software. \url{https://doi.org/10.1016/S0164-1212(03)00015-3}

        \bibitem{Holz06} Holz, Applin, Haberman. (2006). \emph{Research Methods in Computing: What are they, and how should we teach them?}, ITiCSE-WGR: Working Group Reports on ITiCSE on Innovation and Technology in Computer Science Education. \url{https://doi.org/10.1145/1189215.1189180}
        
        \bibitem{Shapiro15} Shapiro. (2015). \emph{Research Methods in Computer Science}, School of Computer Science, University of Manchester. \url{https://studentnet.cs.manchester.ac.uk/pgr/2014/COMP80122/RM.pdf}
    \end{thebibliography}

\end{document}